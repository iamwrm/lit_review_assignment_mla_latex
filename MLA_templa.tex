\documentclass[12pt]{article}

%
%Margin - 1 inch on all sides
%
\usepackage[letterpaper]{geometry}
\usepackage{times}
\geometry{top=1.0in, bottom=1.0in, left=1.0in, right=1.0in}

%
%Doublespacing
%
\usepackage{setspace}
\doublespacing

%
%Rotating tables (e.g. sideways when too long)
%
\usepackage{rotating}


%
%Fancy-header package to modify header/page numbering (insert last name)
%
\usepackage{fancyhdr}
\pagestyle{fancy}
\lhead{} 
\chead{} 
\rhead{Wang \thepage} 
\lfoot{} 
\cfoot{} 
\rfoot{} 
\renewcommand{\headrulewidth}{0pt} 
\renewcommand{\footrulewidth}{0pt} 
%To make sure we actually have header 0.5in away from top edge
%12pt is one-sixth of an inch. Subtract this from 0.5in to get headsep value
\setlength\headsep{0.333in}

%
%Works cited environment
%(to start, use \begin{workscited...}, each entry preceded by \bibent)
% - from Ryan Alcock's MLA style file
%
\newcommand{\bibent}{\noindent \hangindent 40pt}
\newenvironment{workscited}{\newpage \begin{center} Works Cited \end{center}}{\newpage }


%
%Begin document
%
\begin{document}
\begin{flushleft}

%%%%First page name, class, etc
Ren, Wang\\
Professor Cynthia Vagnetti \\
VY200 Section 1\\
\today\\


%%%%Title
\begin{center}
  Communication in Service Industry
\end{center}


%%%%Changes paragraph indentation to 0.5in
\setlength{\parindent}{0.5in}
%%%%Begin body of paper here


%-----Introduction
% Attention grabber
Williams states the dominate status of communication in sale industry through
``The industrial salesperson's most basic activity during the exchange is
communication''(Williams, 29). Winter provides that ``men and women exhibit
different leadership styles and interpersonal communication styles in a variety
of small group situations from student problem-solving situations to industry
and community situations''(Winter, 44), indicating that the property of communicating
differs within different contexts.
Since communication is so significant in business, especially in service
situations, the nature and properties of communication worth investigating in detail.
% The definition of communication
Communication is defined as the imparting or exchanging of information or news.
% verbal and nonverbal
Communication can be classified into verbal one and nonverbal one. Verbal
communication means the exchange of information in speaking of notes passing and
nonverbal communication means the process without spoken or written words. In
fact, it is suggested that ``Nonverbal communication accounts for nearly 70 percent
of all communication ''(Barnum and Wolniansky, 1989).
% communication in service
In service industry, communication means that the service providers receive the
information from the customers about what service they want, either a physical
service like a specific type of coffee or a virtual one like a overall haircut.
In service industry, or to be clearer, a coffee shop the social interaction in
verbal communication is that the barista provides what coffee this shop has and
customers state what they want. The social interaction in nonverbal
communication is that the barista should talk with a smile and make
coffee with enthusiasm so that the customers will be comfortable to pay for the
service they have ordered.
% consequence
If there is a barrier between the service providers and customers about the real
needs of the customers and what this company provides at that time, the
business won't be able to proceed and both sides can't get what they want. Every
company wants to avoid such a lose-lose situation for such a situation will not
only lose money because the service is not undergoing, but also, the company
will suffer a bad reputation for the customers may be disappointed or even angry
about that unsuccessful communication and then tell others not to use the
service from this company.
Without communication, every person is isolated, and one individual's ability
can't make enormous achievements. In addition, by false communication, people
may misunderstand each other, thus becoming isolated or even hostile.

%-----Body      three articles

% art one
Sundaram states the importance of communication for service encounters.
The paper devotes in improving the nonverbal communication in order to improve
customers' evaluation on employees, furthermore expanding business profits. 
The way of communicating information without words is in greatest amount of all
communication. Service employees' nonverbal behaviors contributes a lot to
customers' perceptions of communicators' credibility. Sundaram groups nonverbal
communication into four categories: paralanguage, kinetics, proxemics and
physical appearance. The author then develops a ``conceptional model'' to
explain the mechanism of how nonverbal communication affects customers'
evaluation. Sundaram shows that pleasing nonverbal cues producing positive
affect will result in encouraging customer to connect positive characteristics
with service providers, and vice versa. Although Sundaram clearly indicates 
nonverbal communication's affection on the business, the author lacks a detailed
analysis on nonverbal cues itself. Phutela's article discusses nonverbal cues in
a more specific way.

% art two
Phutela provides that non-verbal communication is defined as a silent form of
communicating with a person or party without any form of speech and in many
cases non-verbal communication have greater effects than verbal communication.
For non-verbal rules differ according to the situations and people, it
can also become obstacle to effective communication.
When verbal and non-verbal communication conflicts, people tend to believe in
non-verbal communication.
Phutela classifies non-verbal communication into four categories and they are
aesthetic communication, physical communication, signs and symbols of
communication.
Phutela regards physical communication as the most used type of communication,
including distance, facial expression, gesture and eye contact.
Then the author provides that non-verbal communication’s effect is to repeat,
check substitute people’s message.
To be more persuasive, Phutela inserts a diagram indicating the enormous amount
of body language making an impression.
After listing tips and abilities for non-verbal communication by two tables,
Phutela leads to a conclusion.
For non-verbal communication is in various forms, an effective one needs the
receiver’s competent interpretation ability, thus acquiring time and practice.
As Sundaram and Phutela together provide the importance of nonverbal 
communication in service business and the nature of nonverbal communication
itself, Keyton gives a study in verbal communication.

% art three
Keyton states that verbal communication in workplace includes four factors:
information sharing, relational maintenance, expressing negative emotion and
organizing communication behaviors. Employers consider oral communication among
the three most valued applied skills, while new graduates are largely deficient
in this property. Keyton provides a way to train verbal communication ability is
to know which behaviors employees use routinely and effectively. To make a
statistics about employees' routine behaviors, the author firstly define the
routine behaviors as the composition of acts, interacts, and double
interacts(Fisher, 1980). Keyton furtherly asserts communication behaviors are
inherently social, useful to engage relationship within a social group and link
micro actions of individuals to macro communication patterns and collective
structures.
Then Keyton uses online survey to collect data, and the result is that
listening, asking questions and discussing are the three most frequently
identified communication behaviors. As a result, many employers gives people who
have such abilities offer or perferm training so that the present employees can
develop those communication abilities.




%-----Conclusion
To sum up, this literature review has gone over the studies about both verbal
and nonverbal communication and its practice in service business, providing an
overall assessment of how important communication in service industry is.
Since people have already been aware of the significance of communication, studies
have been carried out and many service provider company have trained their
employees' abilities of both verbal and nonverbal communication. For some
routine communication, employees are well trained to go over a process of
speaking and gesture. For example, at a coffee shop, what the barista asks you,
in what way, in what smile are all set up by designers who spend their entire
life to improve customer's impression on service. Nobody has compared the
present ``impoved'' employees with the previous one. For business management, improving
communication abilities to achieve more profits is right, but what if employees
are overtrained? Compared to programmed robots, some people may still like the
previous ``more human'' service provider, though he may not be able to smile in
the best way. At present, industrializaition has affected the nature and beauty
of people's original communication.
%%% end of the article main part
\newpage
%%%%Works cited
\begin{workscited}
%% alphabetical order!


  
\bibent
Barnum, Cynthia, and Natasha Wolniansky. ``Taking cues from body language.'' \textit{Management review} 78.6 (1989): 59-61.

\bibent
Fisher, B. Aubrey, and Donald G. Ellis. \textit{Small group decision making: Communication and the group process.} New York: McGraw-Hill, 1980.

\bibent
Keyton, Joann, et al. ``Investigating verbal workplace communication
behaviors.'' \textit{The Journal of Business Communication (1973) } 50.2 (2013): 152-169. 

\bibent
Phutela, Deepika. ``The importance of non-verbal communication.'' \textit{IUP Journal of Soft Skills} 9.4 (2015): 43.

\bibent
Williams, Kaylene C., Rosann L. Spiro, and Leslie M. Fine. ``The customer-salesperson dyad: An interaction/communication model and review.'' \textit{Journal of Personal Selling \& Sales Management} 10.3 (1990): 29-43.

\bibent
Winter, Janet K., Joan C. Neal, and Karen K. Waner. ``How male, female, and mixed-gender groups regard interaction and leadership differences in the business communication course.''  \textit{Business Communication Quarterly} 64.3 (2001): 43-58.



%% alphabetical order!
\end{workscited}

\end{flushleft}
\end{document}
\}