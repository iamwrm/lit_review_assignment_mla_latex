\documentclass[12pt]{article}

%
%Margin - 1 inch on all sides
%
\usepackage[letterpaper]{geometry}
\usepackage{times}
\geometry{top=1.0in, bottom=1.0in, left=1.0in, right=1.0in}

%
%Doublespacing
%
\usepackage{setspace}
\doublespacing

%
%Rotating tables (e.g. sideways when too long)
%
\usepackage{rotating}


%
%Fancy-header package to modify header/page numbering (insert last name)
%
\usepackage{fancyhdr}
\pagestyle{fancy}
\lhead{} 
\chead{} 
\rhead{Wang \thepage} 
\lfoot{} 
\cfoot{} 
\rfoot{} 
\renewcommand{\headrulewidth}{0pt} 
\renewcommand{\footrulewidth}{0pt} 
%To make sure we actually have header 0.5in away from top edge
%12pt is one-sixth of an inch. Subtract this from 0.5in to get headsep value
\setlength\headsep{0.333in}

%
%Works cited environment
%(to start, use \begin{workscited...}, each entry preceded by \bibent)
% - from Ryan Alcock's MLA style file
%
\newcommand{\bibent}{\noindent \hangindent 40pt}
\newenvironment{workscited}{\newpage \begin{center} Works Cited \end{center}}{\newpage }


%
%Begin document
%
\begin{document}
\begin{flushleft}

%%%%First page name, class, etc
Ren, Wang\\
Professor Cynthia Vagnetti \\
VY200 Section 1\\
\today\\


%%%%Title
\begin{center}
  Communication in Service Industry
\end{center}


%%%%Changes paragraph indentation to 0.5in
\setlength{\parindent}{0.5in}
%%%%Begin body of paper here
%Introduction
% Attention grabber
Williams states the dominate status of communication in sale industry through
``The industrial salesperson's most basic activity during the exchange is
communication''(Williams, 29). Winter provides that ``men and women exhibit
different leadership styles and interpersonal communication styles in a variety
of small group situations from student problem-solving situations to industry
and community situations''(Winter, 44), indicating that the property of communicating
differs within different contexts.
Since communication is so significant in business, especially in service
situations, the nature and properties of communication worth investigating in detail.
% The definition of communication
Communication can be defined as the imparting or exchanging of information or news.
% verbal and nonverbal
Communication can be classified into verbal one and nonverbal one. Verbal
communication means the exchange of information in speaking of notes passing and
nonverbal communication means the process without spoken or written words. In
fact, it is suggested that ``Nonverbal communication accounts for nearly 70 percent
of all communication ''(Barnum and Wolniansky, 1989).
% communication in service
In service industry, communication means that the service providers receive the
information from the customers about what service they want, either a physical
service like a specific type of coffee or a virtual one like a overall message.
In service industry, or to be clearer, a coffee shop the social interaction in
verbal communication is that the barista provides what coffee this shop has and
customers state what they want. The social interaction in nonverbal
communication is that the barista should talk with a smile and make
coffee with enthusiasm so that the customers will be comfortable to pay for the
service they have ordered.
% consequence
If there is a barrier between the service providers and customers about the real
needs of the customers and what this company provides at that time, the
business won't be able to proceed and both sides can't get what they want. Every
company wants to avoid such a lose-lose situation for such a situation will not
only lose money because the service is not undergoing, but also, the company
will suffer a bad reputation for the customers may be disappointed or even angry
about that unsuccessful communication and then tell others not to use the
service from this company.



%Phutela states that non-verbal communication is defined as a silent form of
%communicating with a person or party without any form of speech and in many
%cases non-verbal communication have greater effects than verbal communication.
%For non-verbal rules differ according to the situations and people, it
%can also become obstacle to effective communication.
%When verbal and non-verbal communication conflicts, people tend to believe in
%non-verbal communication.
%Phutela classifies non-verbal communication into four categories and they are
%aesthetic communication, physical communication, signs and symbols of
%communication.
%Phutela regards physical communication as the most used type of communication,
%including distance, facial expression, gesture and eye contact.
%Then the author provides that non-verbal communication’s effect is to repeat,
%check substitute people’s message.
%To be more persuasive, Phutela inserts a diagram indicating the enormous amount
%of body language making an impression.
%After that, Phutela emphasizes the importance of non-verbal communication at
%workplace by give examples like handshake, eye contact and dressing.
%After listing tips and abilities for non-verbal communication by two tables,
%Phutela leads to a conclusion.
%For non-verbal communication is in various forms, an effective one needs the
%receiver’s competent interpretation ability, thus acquiring time and practice.



\newpage
%%%%Works cited
\begin{workscited}
%% alphabetical order!



  
\bibent
Barnum, Cynthia, and Natasha Wolniansky. ``Taking cues from body language.'' \textit{Management review} 78.6 (1989): 59-61.
  
\bibent
Williams, Kaylene C., Rosann L. Spiro, and Leslie M. Fine. ``The customer-salesperson dyad: An interaction/communication model and review.'' \textit{Journal of Personal Selling \& Sales Management} 10.3 (1990): 29-43.

\bibent
Winter, Janet K., Joan C. Neal, and Karen K. Waner. ``How male, female, and mixed-gender groups regard interaction and leadership differences in the business communication course.''  \textit{Business Communication Quarterly} 64.3 (2001): 43-58.



%% alphabetical order!
\end{workscited}

\end{flushleft}
\end{document}
\}